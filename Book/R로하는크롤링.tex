% Options for packages loaded elsewhere
\PassOptionsToPackage{unicode}{hyperref}
\PassOptionsToPackage{hyphens}{url}
%
\documentclass[
]{article}
\usepackage{lmodern}
\usepackage{amssymb,amsmath}
\usepackage{ifxetex,ifluatex}
\ifnum 0\ifxetex 1\fi\ifluatex 1\fi=0 % if pdftex
  \usepackage[T1]{fontenc}
  \usepackage[utf8]{inputenc}
  \usepackage{textcomp} % provide euro and other symbols
\else % if luatex or xetex
  \usepackage{unicode-math}
  \defaultfontfeatures{Scale=MatchLowercase}
  \defaultfontfeatures[\rmfamily]{Ligatures=TeX,Scale=1}
\fi
% Use upquote if available, for straight quotes in verbatim environments
\IfFileExists{upquote.sty}{\usepackage{upquote}}{}
\IfFileExists{microtype.sty}{% use microtype if available
  \usepackage[]{microtype}
  \UseMicrotypeSet[protrusion]{basicmath} % disable protrusion for tt fonts
}{}
\makeatletter
\@ifundefined{KOMAClassName}{% if non-KOMA class
  \IfFileExists{parskip.sty}{%
    \usepackage{parskip}
  }{% else
    \setlength{\parindent}{0pt}
    \setlength{\parskip}{6pt plus 2pt minus 1pt}}
}{% if KOMA class
  \KOMAoptions{parskip=half}}
\makeatother
\usepackage{xcolor}
\IfFileExists{xurl.sty}{\usepackage{xurl}}{} % add URL line breaks if available
\IfFileExists{bookmark.sty}{\usepackage{bookmark}}{\usepackage{hyperref}}
\hypersetup{
  pdftitle={R로 하는 크롤링},
  hidelinks,
  pdfcreator={LaTeX via pandoc}}
\urlstyle{same} % disable monospaced font for URLs
\usepackage[margin=1in]{geometry}
\usepackage{color}
\usepackage{fancyvrb}
\newcommand{\VerbBar}{|}
\newcommand{\VERB}{\Verb[commandchars=\\\{\}]}
\DefineVerbatimEnvironment{Highlighting}{Verbatim}{commandchars=\\\{\}}
% Add ',fontsize=\small' for more characters per line
\usepackage{framed}
\definecolor{shadecolor}{RGB}{248,248,248}
\newenvironment{Shaded}{\begin{snugshade}}{\end{snugshade}}
\newcommand{\AlertTok}[1]{\textcolor[rgb]{0.94,0.16,0.16}{#1}}
\newcommand{\AnnotationTok}[1]{\textcolor[rgb]{0.56,0.35,0.01}{\textbf{\textit{#1}}}}
\newcommand{\AttributeTok}[1]{\textcolor[rgb]{0.77,0.63,0.00}{#1}}
\newcommand{\BaseNTok}[1]{\textcolor[rgb]{0.00,0.00,0.81}{#1}}
\newcommand{\BuiltInTok}[1]{#1}
\newcommand{\CharTok}[1]{\textcolor[rgb]{0.31,0.60,0.02}{#1}}
\newcommand{\CommentTok}[1]{\textcolor[rgb]{0.56,0.35,0.01}{\textit{#1}}}
\newcommand{\CommentVarTok}[1]{\textcolor[rgb]{0.56,0.35,0.01}{\textbf{\textit{#1}}}}
\newcommand{\ConstantTok}[1]{\textcolor[rgb]{0.00,0.00,0.00}{#1}}
\newcommand{\ControlFlowTok}[1]{\textcolor[rgb]{0.13,0.29,0.53}{\textbf{#1}}}
\newcommand{\DataTypeTok}[1]{\textcolor[rgb]{0.13,0.29,0.53}{#1}}
\newcommand{\DecValTok}[1]{\textcolor[rgb]{0.00,0.00,0.81}{#1}}
\newcommand{\DocumentationTok}[1]{\textcolor[rgb]{0.56,0.35,0.01}{\textbf{\textit{#1}}}}
\newcommand{\ErrorTok}[1]{\textcolor[rgb]{0.64,0.00,0.00}{\textbf{#1}}}
\newcommand{\ExtensionTok}[1]{#1}
\newcommand{\FloatTok}[1]{\textcolor[rgb]{0.00,0.00,0.81}{#1}}
\newcommand{\FunctionTok}[1]{\textcolor[rgb]{0.00,0.00,0.00}{#1}}
\newcommand{\ImportTok}[1]{#1}
\newcommand{\InformationTok}[1]{\textcolor[rgb]{0.56,0.35,0.01}{\textbf{\textit{#1}}}}
\newcommand{\KeywordTok}[1]{\textcolor[rgb]{0.13,0.29,0.53}{\textbf{#1}}}
\newcommand{\NormalTok}[1]{#1}
\newcommand{\OperatorTok}[1]{\textcolor[rgb]{0.81,0.36,0.00}{\textbf{#1}}}
\newcommand{\OtherTok}[1]{\textcolor[rgb]{0.56,0.35,0.01}{#1}}
\newcommand{\PreprocessorTok}[1]{\textcolor[rgb]{0.56,0.35,0.01}{\textit{#1}}}
\newcommand{\RegionMarkerTok}[1]{#1}
\newcommand{\SpecialCharTok}[1]{\textcolor[rgb]{0.00,0.00,0.00}{#1}}
\newcommand{\SpecialStringTok}[1]{\textcolor[rgb]{0.31,0.60,0.02}{#1}}
\newcommand{\StringTok}[1]{\textcolor[rgb]{0.31,0.60,0.02}{#1}}
\newcommand{\VariableTok}[1]{\textcolor[rgb]{0.00,0.00,0.00}{#1}}
\newcommand{\VerbatimStringTok}[1]{\textcolor[rgb]{0.31,0.60,0.02}{#1}}
\newcommand{\WarningTok}[1]{\textcolor[rgb]{0.56,0.35,0.01}{\textbf{\textit{#1}}}}
\usepackage{graphicx,grffile}
\makeatletter
\def\maxwidth{\ifdim\Gin@nat@width>\linewidth\linewidth\else\Gin@nat@width\fi}
\def\maxheight{\ifdim\Gin@nat@height>\textheight\textheight\else\Gin@nat@height\fi}
\makeatother
% Scale images if necessary, so that they will not overflow the page
% margins by default, and it is still possible to overwrite the defaults
% using explicit options in \includegraphics[width, height, ...]{}
\setkeys{Gin}{width=\maxwidth,height=\maxheight,keepaspectratio}
% Set default figure placement to htbp
\makeatletter
\def\fps@figure{htbp}
\makeatother
\setlength{\emergencystretch}{3em} % prevent overfull lines
\providecommand{\tightlist}{%
  \setlength{\itemsep}{0pt}\setlength{\parskip}{0pt}}
\setcounter{secnumdepth}{-\maxdimen} % remove section numbering

\title{R로 하는 크롤링}
\author{}
\date{\vspace{-2.5em}}

\begin{document}
\maketitle

교보문고 홈페이지에서 종합베스트셀러 1위 제목을 가져오는 것입니다.\\
종합주간베스트 페이지는 교보문고 메인페이지에서 `베스트' 버튼을 클릭하면
접근이 가능합니다.\\
\url{http://www.kyobobook.co.kr/bestSellerNew/bestseller.laf?orderClick=d79}~\\
먼저 위 URL에 있는 html 코드를 가져와야합니다.\\
html 코드를 읽을 때, R에서는 두가지 함수가 사용됩니다.

\begin{enumerate}
\def\labelenumi{\arabic{enumi})}
\tightlist
\item
  GET 함수를 이용하여 교보문고 서버에 내용 요청
\item
  read\_html 함수를 이용하여 html 코드 읽음
\end{enumerate}

R코딩을 해봅시다.\\
GET 함수는 httr 패키지, read\_html 은 rvest 패키지에 들어있는
함수입니다.\\
따라서 먼저 패키지를 설치해주셔야 합니다.

library 함수를 이용하여 패키지를 불러옵니다.

\begin{Shaded}
\begin{Highlighting}[]
\CommentTok{# R 패키지 설치}
\CommentTok{# install.packages("httr")}
\CommentTok{# install.packages("rvest")}

\KeywordTok{library}\NormalTok{(httr)}
\KeywordTok{library}\NormalTok{(rvest)}
\end{Highlighting}
\end{Shaded}

\begin{verbatim}
## Loading required package: xml2
\end{verbatim}

이제 GET함수를 이용하여 서버에 정보를 요청합시다.\\
url이라는 변수에 url주소를 문자열로 저장하였고, 이 변수에 GET 함수를
적용하였습니다.\\
GET 함수에서 요청한 결과를 data에 저장하였습니다.

\begin{Shaded}
\begin{Highlighting}[]
\NormalTok{url =}\StringTok{ 'http://www.kyobobook.co.kr/bestSellerNew/bestseller.laf?orderClick=d79'}
\NormalTok{data =}\StringTok{ }\KeywordTok{GET}\NormalTok{(url)}
\NormalTok{data}
\end{Highlighting}
\end{Shaded}

\begin{verbatim}
## Response [http://www.kyobobook.co.kr/bestSellerNew/bestseller.laf?orderClick=d79]
##   Date: 2020-10-14 08:33
##   Status: 200
##   Content-Type: text/html; charset=EUC-KR
##   Size: 296 kB
## 
## 
## 
## 
## 
## 
## 
## 
## 
## 
## ...
\end{verbatim}

read\_html 함수를 이용하여 html 코드를 읽어봅시다.

\begin{Shaded}
\begin{Highlighting}[]
\NormalTok{my_html=}\KeywordTok{read_html}\NormalTok{(data,}\DataTypeTok{encoding=}\StringTok{'EUC-KR'}\NormalTok{)}
\NormalTok{my_html}
\end{Highlighting}
\end{Shaded}

\begin{verbatim}
## {html_document}
## <html xmlns="http://www.w3.org/1999/xhtml" lang="ko" xml:lang="ko">
## [1] <head>\n<title>교보문고 종합 주간 집계 | 국내도서 | 베스트셀러 - 교보문고</title>\n<!--MS의 최신 웹브 ...
## [2] <body>\n<iframe name="HiddenActionFrame" frameborder="0" width="0" height ...
\end{verbatim}

일단은 strong 태그를 전부 가져와봅시다.

\begin{Shaded}
\begin{Highlighting}[]
\NormalTok{pick1=}\KeywordTok{html_nodes}\NormalTok{(my_html,}\StringTok{'strong'}\NormalTok{)}
\NormalTok{pick1}
\end{Highlighting}
\end{Shaded}

\begin{verbatim}
## {xml_nodeset (166)}
##  [1] <strong>바로콘으로 교보문고 소식 받기</strong>
##  [2] <strong>서울</strong>
##  [3] <strong>수도권</strong>
##  [4] <strong>지방</strong>
##  [5] <strong>자동로그인이 설정되어 있습니다.</strong>
##  [6] <strong>분야 종합</strong>
##  [7] <strong><a href="javascript:_go_targetPage('1')">1</a></strong>
##  [8] <strong class="rank">1</strong>
##  [9] <strong>보건교사 안은영(특별판)(양장본 HardCover)</strong>
## [10] <strong class="book_price">12,600원</strong>
## [11] <strong>10</strong>
## [12] <strong>5</strong>
## [13] <strong class="blue">내일(15일,목)</strong>
## [14] <strong class="blue">  도착 예정</strong>
## [15] <strong class="rank">2</strong>
## [16] <strong>달러구트 꿈 백화점</strong>
## [17] <strong class="book_price">12,420원</strong>
## [18] <strong>10</strong>
## [19] <strong>5</strong>
## [20] <strong class="ebook_price">9,100원</strong>
## ...
\end{verbatim}

\begin{Shaded}
\begin{Highlighting}[]
\KeywordTok{class}\NormalTok{(pick1)}
\end{Highlighting}
\end{Shaded}

\begin{verbatim}
## [1] "xml_nodeset"
\end{verbatim}

\begin{Shaded}
\begin{Highlighting}[]
\KeywordTok{length}\NormalTok{(pick1)}
\end{Highlighting}
\end{Shaded}

\begin{verbatim}
## [1] 166
\end{verbatim}

1위 도서가 9번째 줄에 1위 도서 제목이 있습니다. 인덱싱을 통해
접근해봅시다 .

\begin{Shaded}
\begin{Highlighting}[]
\NormalTok{pick1[}\DecValTok{9}\NormalTok{]}
\end{Highlighting}
\end{Shaded}

\begin{verbatim}
## {xml_nodeset (1)}
## [1] <strong>보건교사 안은영(특별판)(양장본 HardCover)</strong>
\end{verbatim}

이제 태그의 내용만 가져오면 됩니다. html\_text 함수를 사용합니다.

\begin{Shaded}
\begin{Highlighting}[]
\NormalTok{result=}\KeywordTok{html_text}\NormalTok{(pick1[}\DecValTok{9}\NormalTok{])}
\NormalTok{result}
\end{Highlighting}
\end{Shaded}

\begin{verbatim}
## [1] "보건교사 안은영(특별판)(양장본 HardCover)"
\end{verbatim}

\begin{Shaded}
\begin{Highlighting}[]
\CommentTok{# 네이버 실시간 검색어 읽기}
\NormalTok{url =}\StringTok{ "https://datalab.naver.com/keyword/realtimeList.naver"}
\NormalTok{data =}\StringTok{ }\KeywordTok{GET}\NormalTok{(url)}
\NormalTok{data}
\end{Highlighting}
\end{Shaded}

\begin{verbatim}
## Response [https://datalab.naver.com/keyword/realtimeList.naver]
##   Date: 2020-10-14 08:33
##   Status: 200
##   Content-Type: text/html;charset=UTF-8
##   Size: 44.5 kB
## <!DOCTYPE html>
## <html lang="ko">
## <head>
##     <meta charset="utf-8">
##     <meta http-equiv="X-UA-Compatible" content="IE=edge">
##     <meta name="viewport" content="width=1200">
##     <title>급상승검색어 : 네이버 데이터랩</title>
##     <link rel="stylesheet" type="text/css" href="https://ssl.pstatic.net/stat...
##     <link rel="stylesheet" type="text/css" href="https://ssl.pstatic.net/stat...
## 
## ...
\end{verbatim}

\begin{Shaded}
\begin{Highlighting}[]
\NormalTok{my_html=}\KeywordTok{read_html}\NormalTok{(data,}\DataTypeTok{encoding=}\StringTok{'UTF-8'}\NormalTok{)}
\NormalTok{my_html}
\end{Highlighting}
\end{Shaded}

\begin{verbatim}
## {html_document}
## <html lang="ko">
## [1] <head>\n<meta http-equiv="Content-Type" content="text/html; charset=UTF-8 ...
## [2] <body>\n\n\n<div id="wrap" class="wrap">\n    \n    \n    <div id="header ...
\end{verbatim}

\begin{Shaded}
\begin{Highlighting}[]
\NormalTok{pick1=}\KeywordTok{html_nodes}\NormalTok{(my_html,}\StringTok{'ul.ranking_list'}\NormalTok{)}
\NormalTok{pick1}
\end{Highlighting}
\end{Shaded}

\begin{verbatim}
## {xml_nodeset (2)}
## [1] <ul class="ranking_list">\n<li class="ranking_item">\n                    ...
## [2] <ul class="ranking_list">\n<li class="ranking_item">\n                    ...
\end{verbatim}

\begin{Shaded}
\begin{Highlighting}[]
\NormalTok{pick1[}\DecValTok{1}\NormalTok{]}
\end{Highlighting}
\end{Shaded}

\begin{verbatim}
## {xml_nodeset (1)}
## [1] <ul class="ranking_list">\n<li class="ranking_item">\n                    ...
\end{verbatim}

\begin{Shaded}
\begin{Highlighting}[]
\NormalTok{result=}\KeywordTok{html_text}\NormalTok{(pick1[}\DecValTok{1}\NormalTok{], }\DataTypeTok{trim =}\NormalTok{ T)}
\NormalTok{result}
\end{Highlighting}
\end{Shaded}

\begin{verbatim}
## [1] "1\n                            \n                            김선아\n                            \n                            \n                            \n                        \n                        \n                            \n                            \n                            2\n                            \n                            박혜수\n                            \n                            \n                            \n                        \n                        \n                            \n                            \n                            3\n                            \n                            디어엠\n                            \n                            \n                            \n                        \n                        \n                            \n                            \n                            4\n                            \n                            김새론\n                            \n                            \n                            \n                        \n                        \n                            \n                            \n                            5\n                            \n                            정의선\n                            \n                            \n                            \n                        \n                        \n                            \n                            \n                            6\n                            \n                            해뜨락요양병원\n                            \n                            \n                            \n                        \n                        \n                            \n                            \n                            7\n                            \n                            영화 디파티드\n                                \n                                    디파티드\n                                \n                            \n                            \n                            \n                        \n                        \n                            \n                            \n                            8\n                            \n                            김선동\n                            \n                            \n                            \n                        \n                        \n                            \n                            \n                            9\n                            \n                            아이폰 12 pro\n                            \n                            \n                            \n                        \n                        \n                            \n                            \n                            10\n                            \n                            설리"
\end{verbatim}

\begin{Shaded}
\begin{Highlighting}[]
\KeywordTok{library}\NormalTok{(stringr)}
\NormalTok{result =}\StringTok{ }\KeywordTok{str_replace_all}\NormalTok{(result, }\StringTok{'}\CharTok{\textbackslash{}n}\StringTok{'}\NormalTok{, }\StringTok{''}\NormalTok{)}
\NormalTok{result}
\end{Highlighting}
\end{Shaded}

\begin{verbatim}
## [1] "1                                                        김선아                                                                                                                                                                                                                        2                                                        박혜수                                                                                                                                                                                                                        3                                                        디어엠                                                                                                                                                                                                                        4                                                        김새론                                                                                                                                                                                                                        5                                                        정의선                                                                                                                                                                                                                        6                                                        해뜨락요양병원                                                                                                                                                                                                                        7                                                        영화 디파티드                                                                    디파티드                                                                                                                                                                                                                                                        8                                                        김선동                                                                                                                                                                                                                        9                                                        아이폰 12 pro                                                                                                                                                                                                                        10                                                        설리"
\end{verbatim}

\begin{Shaded}
\begin{Highlighting}[]
\CommentTok{# 중간의 공백 제거하기}
\NormalTok{result =}\StringTok{ }\KeywordTok{str_squish}\NormalTok{(result)}
\NormalTok{result}
\end{Highlighting}
\end{Shaded}

\begin{verbatim}
## [1] "1 김선아 2 박혜수 3 디어엠 4 김새론 5 정의선 6 해뜨락요양병원 7 영화 디파티드 디파티드 8 김선동 9 아이폰 12 pro 10 설리"
\end{verbatim}

\begin{Shaded}
\begin{Highlighting}[]
\NormalTok{result=}\KeywordTok{html_text}\NormalTok{(pick1[}\DecValTok{2}\NormalTok{])}
\NormalTok{result =}\StringTok{ }\KeywordTok{str_replace_all}\NormalTok{(result, }\StringTok{'}\CharTok{\textbackslash{}n}\StringTok{'}\NormalTok{, }\StringTok{''}\NormalTok{)}
\NormalTok{result =}\StringTok{ }\KeywordTok{str_squish}\NormalTok{(result)}
\NormalTok{result}
\end{Highlighting}
\end{Shaded}

\begin{verbatim}
## [1] "11 호날두 12 호날두 코로나 13 권영세 14 김선아 설리 15 스파이더맨 3 16 옵티머스사건 17 부산해뜨락요양병원 18 정몽구 19 10월 14일 20 시스웍"
\end{verbatim}

\begin{Shaded}
\begin{Highlighting}[]
\CommentTok{#라이브러리 불러오기}
\KeywordTok{library}\NormalTok{(httr)}
\KeywordTok{library}\NormalTok{(rvest)}
\CommentTok{#GET 함수로 서버에 정보 요청하기}
\NormalTok{url =}\StringTok{ 'https://www.melon.com/chart/'}
\NormalTok{get_url =}\StringTok{ }\KeywordTok{GET}\NormalTok{(url)}
\CommentTok{#read_html 함수로 html 코드 읽기}
\NormalTok{my_html=}\KeywordTok{read_html}\NormalTok{(get_url,}\DataTypeTok{encoding=}\StringTok{'utf-8'}\NormalTok{)}
\CommentTok{#ellipsis rank01 클래스만 추출}
\NormalTok{pick1=}\KeywordTok{html_nodes}\NormalTok{(my_html,}\StringTok{'.ellipsis.rank01'}\NormalTok{)}
\CommentTok{#a 태그만 추출}
\NormalTok{pick2=}\KeywordTok{html_nodes}\NormalTok{(pick1,}\StringTok{'a'}\NormalTok{)}
\CommentTok{#텍스트 추출}
\NormalTok{pick3=}\KeywordTok{html_text}\NormalTok{(pick1,}\DataTypeTok{trim=}\OtherTok{TRUE}\NormalTok{)}
\CommentTok{#top10순위}
\KeywordTok{print}\NormalTok{(pick3[}\DecValTok{1}\OperatorTok{:}\DecValTok{10}\NormalTok{])}
\end{Highlighting}
\end{Shaded}

\begin{verbatim}
##  [1] "DON'T TOUCH ME"                                
##  [2] "Dynamite"                                      
##  [3] "Lovesick Girls"                                
##  [4] "취기를 빌려 (취향저격 그녀 X 산들)"            
##  [5] "When We Disco (Duet with 선미)"                
##  [6] "눈누난나 (NUNU NANA)"                          
##  [7] "오래된 노래"                                   
##  [8] "내 마음이 움찔했던 순간 (취향저격 그녀 X 규현)"
##  [9] "Savage Love (Laxed - Siren Beat) (BTS Remix)"  
## [10] "마리아 (Maria)"
\end{verbatim}

\end{document}
